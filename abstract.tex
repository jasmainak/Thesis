\begin{changemargin}{-2.5cm}{-1.2cm} %XXX book!!!  cf definition.tex a la fin pour options


\begin{abstract}


\begin{center}
{\Large \textbf{Apprentissage Automatique et Extrêmes pour la Détection d'Anomalies\\}}
\textbf{Nicolas Goix}
\end{center}


{\setstretch{1.0}

\paragraph{RESUME}

{\small
La détection d'anomalies, d'outliers ou de nouveauté n'est pas seulement une étape utile de pré-traitement des données pour entrainer un algorithme d'apprentissage statistique. C'est aussi une composante importante d'une grande variété d'applications concrètes, allant de la finance, de l'assurance à la biologie computationnelle en passant par la santé, les télécommunications ou les sciences environnementales. La détection d'anomalie est aussi de plus en plus utile au monde moderne, où il est necessaire de surveiller et de diagnostiquer un nombre croissant de systèmes autonomes. La recherche en détection d'anomalies inclut la création d'algorithmes efficaces accompagnée d'une étude théorique, mais pose aussi la question de l'évaluation de tels algorithmes, particulièrement lorsqu'on ne dispose pas de données labellisées -- comme dans une multitude de contextes industriels.
%
En d'autres termes, l'élaboration du modèle et son étude théorique, mais aussi la sélection du modèle.

Dans cette thèse, nous abordons ces deux aspects. En pratique, un algorithme de détection d'anomalie retourne une \emph{fonction de score} à valeurs réelles définie sur l'espace des données de manière à quantifier l'anormalitédes observations.
%
Tout d'abord, nous introduisons un critère alternatif au critère masse-volume existant, pour mesurer les performances d'une fonction de score.
%
Ce critère, appelé \emph{critère d'excès de masse}, à pour but la construction de fonctions de score \emph{via} la minimisation du risque empirique.

La seconde partie de ce travail porte sur les régions \emph{extrêmes}, qui sont d'un interêt particulier en détection d'anomalie. En particulier, des outils probabilistes issues de la théorie des valeurs extrêmes (multivariées), comme la STDF (stable tail dependence function) et la mesure angulaire, peuvent être combinés avec une approche plus classique de détection d'anomalie afin de gagner en précision sur ces régions extrêmes.
Des bornes non-asymptotiques sont établies pour l'estimation de la STDF, cette dernière caractérisant la structure de dépendence dans les extrêmes.
%
Une méthode statistique produisant une représentation (possiblement parcimonieuse) de la structure de dépendence est ensuite dérivée de l'estimation non-paramétrique de la mesure angulaire restreinte à un ensemble représentatif de directions. Cette représentation peut être utilisée pour produire une fonction de score précise sur les régions extrêmes. Des bornes non-asymptotiques attestant de la qualité de l'estimation sont établies.
%

La dernière partie de ce travail est essentiellement de nature heuristique. D'un point de vue sélection de modèle, nous étudions empiriquement l'usage des courbes masse-volume et d'excès de masse (en l'absence de données labellisées). Comme ces courbes ne peuvent généralement pas être estimées avec qualité en grande dimension, une méthode basée sur le sous-échantillonage de features est aussi developpée et testée, étendant ainsi l'usage de ces deux critères à des jeux de données de grande dimension.

Du point de vue élaboration de modèle, un algorithme efficace basé sur les forêts aléatoires et produisant des fonctions de score précises est proposé. Cet algorithme se base sur une extention naturelle des critères de split standards au cas one-class, et donne des performances remarquables selon une benchmark incluant une multitude d'algorithme de détection d'anomalie utilisés dans l'industrie.
%
}

\textbf{MOTS-CLEFS:} Détection d'anomalies, extrêmes multivariés, forêts aléatoires, sélection de modèles non-supervisé.

\paragraph{ABSTRACT}

{\small

Anomaly, outlier or novelty detection is not only a useful preprocessing step for training machine learning algorithms. It is also a crucial component of many real-world applications, from various fields like finance, insurance, telecommunication, computational biology, health or environmental sciences. Anomaly detection is also more and more relevant in the modern world, as an increasing number of autonomous systems need to be monitored and diagnosed. % -- \eg~with the rise of Internet-of-Things.
%
Important research areas in anomaly detection include the design of efficient algorithms and their theoretical study %(derived from a mathematical theory as well as heuristic-based ones), 
%the theoretical study of existing algorithms lacking from such guaranties, 
but also the evaluation of such algorithms, in particular when no labeled data is available -- as in lots of industrial setups. 
In other words, model design and study, and model selection.

In this thesis, we focus on both of these aspects. 
In practice, anomaly detection algorithms output a real valued \emph{scoring function} on the feature space so as to quantify to which extent observations should be considered as abnormal.
%
We first propose a criterion for measuring the performance of scoring functions, alternative to the existing \emph{mass-volume curve}. %-- function outputed by anomaly detection algorithms and measuring the supposed degree of abnormality of the observations.
This criterion, referred to as the \emph{excess-mass curve}, aims at building scoring functions \emph{via} empirical risk minimization.


The second part of this work focuses on \emph{extreme} regions, which are of particular interest in anomaly detection. In particular, probabilistic tools borrowed from (multivariate) extreme value theory, such as the stable tail dependence function (STDF) and the angular measure, can be combined with classical anomaly detection approaches to gain in accuracy on such extreme regions.
%
%Advances in multivariate EVT are brought by
We provide non-asymptotic bounds for the estimation of the STDF, which characterizes the extreme dependence structure.
%
A statistical method that produces a (possibly sparse) representation of the extreme dependence structure is then derived from a non-parametric estimation of the angular measure on representative sets of directions. This representation can be used to produce a scoring function accurate on extremes regions. Non-asymptotic bounds to assess the accuracy of the estimation procedure are established.

The last part of this work is essentially of heuristic nature. %compiles some limitations of previous parts, and proposes two heuristics, one from the model selection / algorithms evaluation point of view, the other from the algorithms design. %and proposes solutions to the two major drawbacks raised by the Excess-Mass and Mass-Volume curves: poor performance in practice
%
% From the model selection point of view, no empirical study has been made on Excess-Mass and Mass-Volume curves for evaluating anomaly detection algorithms without using any labels. Besides, these curves can only been estimated in small dimensions as they involve some volume computation. 
% and no study has been done on their capacity to discriminate between scoring functions with respect to ROC and PR curves when labels are used.
%If default parameters usually work well for the EVT-based scoring function we promote next, accurate
From the model selection viewpoint, an empirical study for the use of excess-mass and mass-volume curves as evaluation criteria (in the absence of labeled data) is derived. 
 As these curves generally cannot be well estimated in large dimension,
a methodology based on feature sub-sampling and aggregating is also described and tested, extending the use of these criteria to high-dimensional datasets. %and solving major drawbacks inherent to standard EM and MV curves. 
%
From the model design viewpoint,
an efficient algorithm based on random forests producing accurate scoring functions is proposed. It builds on a natural extension of standard splitting criteria to the one-class setting, and competes well according to an extensive benchmark which includes many state-of-the-art anomaly detection algorithms, commonly used in industrial setups.
 %This structural generalizion of random forests to one-class classification produces competitive scoring functions, according to an extensive benchmark which includes many state-of-the-art anomaly detection algorithms commonly used in industrial setups.

\textbf{KEYWORDS:} Anomaly detection, multivariate extremes, random forests, unsupervised model selection
}

}

\end{abstract}

\end{changemargin}
